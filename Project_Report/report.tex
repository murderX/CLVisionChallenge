\documentclass{article}

% if you need to pass options to natbib, use, e.g.:
%     \PassOptionsToPackage{numbers, compress}{natbib}
% before loading neurips_2018

% ready for submission
% \usepackage{neurips_2018}

% to compile a preprint version, e.g., for submission to arXiv, add add the
% [preprint] option:
%     \usepackage[preprint]{neurips_2018}

% to compile a camera-ready version, add the [final] option, e.g.:
     %\usepackage[final]{neurips_2018}

% to avoid loading the natbib package, add option nonatbib:
\usepackage[final, nonatbib]{neurips_2018}

\usepackage[utf8]{inputenc} % allow utf-8 input
\usepackage[T1]{fontenc}    % use 8-bit T1 fonts
\usepackage{hyperref}       % hyperlinks
\usepackage{url}            % simple URL typesetting
\usepackage{booktabs}       % professional-quality tables
\usepackage{amsfonts}       % blackboard math symbols
\usepackage{nicefrac}       % compact symbols for 1/2, etc.
\usepackage{microtype}      % microtypography
\usepackage[authoryear]{natbib}


\title{Continual Learning on CLVision-Challenge}

% The \author macro works with any number of authors. There are two commands
% used to separate the names and addresses of multiple authors: \And and \AND.
%
% Using \And between authors leaves it to LaTeX to determine where to break the
% lines. Using \AND forces a line break at that point. So, if LaTeX puts 3 of 4
% authors names on the first line, and the last on the second line, try using
% \AND instead of \And before the third author name.

\author{%
  Haoran Zhu, Hua Sun \\
  Department of Electrical and Computer Engineering\\
  New York University\\
  \texttt{\{xxxxxx, hs4062\}@nyu.edu} \\
  % examples of more authors
  % \And
  % Coauthor \\
  % Affiliation \\
  % Address \\
  % \texttt{email} \\
  % \AND
  % Coauthor \\
  % Affiliation \\
  % Address \\
  % \texttt{email} \\
  % \And
  % Coauthor \\
  % Affiliation \\
  % Address \\
  % \texttt{email} \\
  % \And
  % Coauthor \\
  % Affiliation \\
  % Address \\
  % \texttt{email} \\
}

\begin{document}
% \nipsfinalcopy is no longer used

\maketitle

\begin{abstract}
  PlaceHolder
\end{abstract}

\section{Introduction}
In today's Deep Learning scheme, the presumption of the training is that all instances of the classes are available, the model will iterate the dataset per epoch to learn the knowledge. This means if there are new instances added to the dataset, we have to retrain the whole model on the whole dataset plus the new instances. This has caused trouble in some scenarios. Specifically, in image classification problems, the pretrained model often encounters new objects or the dataset can be expanded. However, the existence of \textit{Catastrophic Forgetting}\cite{mccloskey1989catastrophic}, i.e. the newly learned parameters will shift the old parameters and weaken its performance on old task, has brought challenge to this problem. This phenomenon can greatly degrade the performance of the model or even totally rewrite the model's parameters, causing the old knowledge being totally forgot. In before, in face of such case, one have to choose either retrain the model on entire dataset or just tolerate such degrading issues. To overcome the dilemma, the concept of continual learning(Lifelong Learning\cite{thrun1995lifelong}) has emerged. In continual learning, it is required that the model must have not only the ability to acquire new knowledge, but also prevent the novel input to overwhelm the original data. 

Continual learning is close to the concept of online learning. But in online learning setting, there will be some dependency on the previous data, while in continual learning, we do not want to access the original dataset because it is costly and time consuming. The focus of continual learning is not only on maintaining the accuracy, but also on training efficiently. Currently, there are three main approaches to apply continual learning,
\begin{enumerate}
\item \textbf{Retraining}
Totally retraining the old parameters $\theta_o$ on the new task to obtain the new model $\theta_n$, and then apply regularization to prevent the degrading problem on old task. The typical work in this approach are \textit{Learning without Forgetting}(LwF)\cite{li2017learning} and  Elastic Weight Consolidation(EWC)\cite{kirkpatrick2017overcoming}. 
\item \textbf{Expansion}
Non-Retraining but instead expanding the network. Whenever encounter the new task, just freeze the old weight and expand the network. One of the typical work is \textit{Progressive Network}\cite{rusu2016progressive}. Another variation of this work is masking, i.e. they focus not only on the weigth itsely, but also considered masking some of the unimportant weights for the new task $t$. Packenet\cite{mallya2018packnet} and Piggyback\cite{mallya2018piggyback} adopt this approach.
\item \textbf{Partial Retraining with Expansion}
Another approach is selectively retraining the old network, expanding the capacity when necessary and train dynamically on the optimal solution. One of the outstanding work in this approach is \textit{Dynamically Expandable Networks}(DEN)\cite{yoon2017lifelong}.
\end{enumerate}

In this project, we focus on the \textit{CLVision Challenge}, i.e. Continual Learning in Computer Vision, a problem set published by CVPR 2020 Workshop to solve the image classification problem in continual learning. We have reproduced and refined several famous approaches, including \textit{EWC}, \textit{Piggyback} and \textit{DEN}(?), and tried out them on the CLVision dataset, then we evaluate the performance and analyze the potential improvements that can be done. 
\section{Related Work}
\section{Dataset and Experiment Design}
\section{Evaluation and Analyze}
\section{Conclusion}

\subsection{Footnotes}

Footnotes should be used sparingly.  If you do require a footnote, indicate
footnotes with a number\footnote{Sample of the first footnote.} in the
text. Place the footnotes at the bottom of the page on which they appear.
Precede the footnote with a horizontal rule of 2~inches (12~picas).

Note that footnotes are properly typeset \emph{after} punctuation
marks.\footnote{As in this example.}

\subsection{Figures}

\begin{figure}
  \centering
  \fbox{\rule[-.5cm]{0cm}{4cm} \rule[-.5cm]{4cm}{0cm}}
  \caption{Sample figure caption.}
\end{figure}

All artwork must be neat, clean, and legible. Lines should be dark enough for
purposes of reproduction. The figure number and caption always appear after the
figure. Place one line space before the figure caption and one line space after
the figure. The figure caption should be lower case (except for first word and
proper nouns); figures are numbered consecutively.

You may use color figures.  However, it is best for the figure captions and the
paper body to be legible if the paper is printed in either black/white or in
color.

\subsection{Tables}

All tables must be centered, neat, clean and legible.  The table number and
title always appear before the table.  See Table~\ref{sample-table}.

Place one line space before the table title, one line space after the
table title, and one line space after the table. The table title must
be lower case (except for first word and proper nouns); tables are
numbered consecutively.

Note that publication-quality tables \emph{do not contain vertical rules.} We
strongly suggest the use of the \verb+booktabs+ package, which allows for
typesetting high-quality, professional tables:
\begin{center}
  \url{https://www.ctan.org/pkg/booktabs}
\end{center}
This package was used to typeset Table~\ref{sample-table}.

\begin{table}
  \caption{Sample table title}
  \label{sample-table}
  \centering
  \begin{tabular}{lll}
    \toprule
    \multicolumn{2}{c}{Part}                   \\
    \cmidrule(r){1-2}
    Name     & Description     & Size ($\mu$m) \\
    \midrule
    Dendrite & Input terminal  & $\sim$100     \\
    Axon     & Output terminal & $\sim$10      \\
    Soma     & Cell body       & up to $10^6$  \\
    \bottomrule
  \end{tabular}
\end{table}

\bibliographystyle{acm}
\bibliography{refer}


\end{document}